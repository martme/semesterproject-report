\chapter{Background}\label{chp:background} 
In this chapter we will look at how side channels have been used earlier, and look at current research on the field of low-bandwidth acoustic cryptanalysis.


\section{Side-Channels}\label{chp2:sec:side_channel}
Computer systems can in theory achieve security, as code can be secure from a mathematical perspective.
However, these systems are realized in a physical form which introduces unintended channels where information is leaked.
These channels include electricity consumption, time taken to compute and electromagnetic emanations. 
All of these have to be taken into consideration when transforming a system from the theoretical model to a physical realization, as even computer code need to be executed on a physical circuit.
Failure to consider this information leakage can enable an attacker to break the systems, e.g. by extracting cryptographic keys.

\subsection{Acoustic Cryptanalysis of Mechanical Systems}\label{chp2:subsec:acoustic_cryptanalysis}
The acoustic side channel has been known as a potential attack vector for mechanical systems since the second world war, where covert microphones were used to find the keys of mechanical cipher machines as the code was renewed every day, producing a mechanical click sound, enabling the eavesdropper to find yesterdays key\todo{Find source and cite!}

In 2005, some Asonov et al. were able to reconstruct what was typed on a keyboard by analyzing the acoustic emanations, using a neural network~\cite{DBLP:conf/sp/AsonovA04}. 
They also showed that phones and ATMs with mechanical keyboards can be vulnerable to this king of acoustic side channel attacks.

\subsection{Timing Attacks}\label{chp2:subsec:timing_attacks}
The devastating result of not taking side channels into consideration in cryptographic implementations was shown by Paul Kocher in 1996, when he presented his findings in how measuring the amount of time required by an operation enabled an attacker to break cryptosystems such as RSA and Diffie-Hellman~\cite{DBLP:conf/crypto/Kocher96}.

In 2011, Brumley et al. published a paper describing a timing attack vulnerability in OpenSSL allowing an attacker to ultimately extract private keys~\cite{DBLP:conf/esorics/BrumleyT11}.
This is a display of how known side channels can be hard to silence.


\subsection{Power Analysis}\label{chp2:subsec:power_analysis}
Electronic devices can leak information through their power consumption, and this was proven by Paul Kocher in 1999~\cite{DBLP:conf/crypto/KocherJJ99}.
The power traces used in power analysis contain much more information than the timing information utilized in timing attacks, and Kocher demonstrated that a single power trace can be enough to read out a full RSA key, with a naive implementation of the cryptographic algorithm using \gls{SPA}.
These kinds of attacks can be empowered by using statistical models, using \gls{DPA}.


\subsection{Electromagnetic Emanation Analysis}\label{chp2:subsec:electromagnetic_attacks}
An adversary can, if located within range, intercept electromagnetic signals emitted by all electronic devices. 
It is not possible to detect this attack, and also hard to prevent. 
One of the known attacks is extracting the electromagnetic signals from a computer monitor to reproduce the screen output.
These signals originate from the electron gun that manipulates each pixel~\cite{url:tempest_sans}. 
The stronger a pixel is, the stronger the signal.
PGP include a secure viewer that prevents these exact high frequency signals by making softened edges in the fonts.
However, it is said to be very difficult, thus expensive to extract the signals emitted by a computer monitor. 


\section{Related Work}\label{chp2:sec:related_work}
The acoustic side channel is the basis for this paper, and the research done in the area of acoustic emanations from a computer is limited.

\subsection{Low-Frequency Acoustic Cryptanalysis}\label{chp2:subsec:original_research}
We base our project on the work of Daniel Genkin, Adi Shamir and Eran Tromer.
The research they have done is the RSA Key Extraction via Low-Bandwidth Acoustic Cryptanalysis~\cite{DBLP:conf/crypto/GenkinST14}.
Their claim is that computers produce low frequency acoustic emanations from their electronic components, possibly caused by vibrations in transistors or electronic components.
The frequency range of the allegedly viable emanations lie in the sub 100kHz range.
Genkin et al. were able to extract a 4096-bit RSA key using this side channel and exploiting a vulnerability in an implementation of RSA.
The claim that the acoustic emanations in this frequency range from a computer system carries information about the state of the computer, is new.
Earlier, acoustic cryptanalysis has been limited to mechanical systems, hence the discovery of this attack vector may open for a whole new range of new attacks.
