\chapter{Background}
\label{chp:background} 

% Present original research and earlier famous side channel attacks 
%    - Early similar attacks on mechanical systems
%    - Paul Coucher papers for perspective
%    - Original research
%    - 
%    - 
Side-channel attacks is not the kind of standard attack like brute-force attack where you try every single combination to crack a cryptographic algorithm, or like a theoretical weakness that you can take advatage of in cryptographic implementations. 
Side-channel attack exploits the physical implementation to gain information about a cryptosystem.
Characteristics a side-channel attack might exploit can be the power consumption, timing information, electromagnetic radiation, information obtained from a storage after deletion or even sound leakage from the target machine. 

Covert-channel.. 
http://en.wikipedia.org/wiki/Side\_channel\_attack


\section{Related work}\label{sec:related_work}

\subsection{Acoustic cryptanalysis}\label{sec:acoustic_cryptanalysis}

Here goes related work to aucustic cryptanlysis

Sarah Yang: Researchers recover typed text using audio recording of keystrokes 
http://www.berkeley.edu/news/media/releases/2005/09/14\_key.shtml


\subsection{Data remanence}\label{sec:data_remanence}

Here goes related work to data remanence

http://en.wikipedia.org/wiki/Data\_remanence

Sergei Skorobogatov: Data Remanence in Flash Memory Devices
https://www.cl.cam.ac.uk/~sps32/DataRem\_CHES2005.pdf

\subsection{Differential fault analysis}\label{sec:differential_fault_analysis}

Here goes related work to differential fault analysis

Guo, et al (2012): Invariance-based concurrent error detection for advanced encryption standard
http://dl.acm.org/citation.cfm?id=2228463

Giraud (2003): DFA on AES
http://citeseerx.ist.psu.edu/viewdoc/summary?doi=10.1.1.12.1030

\subsection{Electromagnetic attacks}\label{sec:electromagnetic_attacks}

Here goes related work to electromagnetic attacks

NSA: TEMPEST (code name for project)
http://en.wikipedia.org/wiki/Tempest\_\%28codename\%29

\subsection{Power analysis}\label{sec:power_analysis}

Here goes related work to power analysis

SPA vs DPA
Simple power analysis (SPA) involves visually interpreting power traces, or graphs of electrical activity over time. 
Differential power analysis (DPA) is a more advanced form of power analysis which can allow an attacker to compute the intermediate values within cryptographic computations by statistically analyzing data collected from multiple cryptographic operations.

Paul Kocher: Differential power analysis
http://www.cryptography.com/public/pdf/DPA.pdf
\subsection{Timing attacks}\label{sec:timing_attacks}

Here goes related work to timing attacks

D. Boneh and D. Brumley: Remote timing attacks are practical:
http://crypto.stanford.edu/~dabo/abstracts/ssl-timing.html

Paul Kocher: Timing attacks on implementations of Diffie-Hellman, RSA, DSS and other systems:
http://www.cryptography.com/public/pdf/TimingAttacks.pdf

\section{Original research - Acoustic Cryptanalysis}\label{sec:original_research}

Tromer, et al: RSA Key Extraction via Low-Bandwidth Acoustic Cryptanalysis
http://cs.tau.ac.il/~tromer/acoustic/

\subsection{Lab setup}\label{sec:lab_setup}
\subsection{Acustic leakage}\label{sec:acustic_leakage}
\subsection{RSA key extraction}\label{sec:rsa_key_extraction}
