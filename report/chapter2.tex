\chapter{Background}
\label{chp:background} 

% Present original research and earlier famous side channel attacks 
%    - Early similar attacks on mechanical systems
%    - Paul Coucher papers for perspective
%    - Original research
%    - 
%    - 
Side-channel attacks is not the kind of standard attack like brute-force attack where you try every single combination to crack a cryptographic algorithm, or like a theoretical weakness that you can take advatage of in cryptographic implementations. 
Side-channel attack exploits the physical implementation to gain information about a cryptosystem.
Characteristics a side-channel attack might exploit can be the power consumption, timing information, electromagnetic radiation, information obtained from a storage after deletion or even sound leakage from the target machine.

Covert-channel attack allows the adversary to communicate with two objects in a physical matter that is not supposed to be allowed. An attack like this is useful when the adversary is able to get inside a computer, but not able to communicate with others what he finds. 
If the adversary knows about a channel, he can use this to transfer the knowledge he has learned.
This channel can be a hardware device that is apparently leaking sound in a frequency that humans cannot hear. \cite{wiki_side_channel}


\section{Related work}\label{sec:related_work}

\subsection{Acoustic cryptanalysis}\label{sec:acoustic_cryptanalysis}

Acoustic cryptanalysis\cite{wiki_acoustic} is done by listening to sound that is emitted by computers. 
Computers leaks sound from Hz to several hundred kHz. 
\todo{Write something more.. general stuff about acoustic cryptanalysis}

But there are other ways to do acoustic cryptanalysis. In 2005, some researchers at Berkeley did a research\cite{keystrokes} on recognizing keystrokes on a keyboard by sound. 
This is possible because each key has its own distinct sound. 
After separating the different keystrokes, they used a statistical frequency method to find out which key belonged to which letter. 
To determine the letters, they only had to listen for 10 minutes, i.e. on a typical user that types about 300 characters each minute. 
One method to take advantage of this attack is to create a mobilephone application that enables the microphone and analyses the keystrokes. 

Its also proven that phones and ATMs with mechanical keyboards can be vulnerable for acoustic attacks\cite{KybdEmanation}.

In 2014, a research team at the Tel Aviv University was able to do a full RSA Key (4096-bit) extraction using acoustic cryptanalysis. 
This research explained in detail at section \ref{sec:original_research}.

\subsection{Data remanence}\label{sec:data_remanence}

Data remanence is remaining representation if digital bits after attempts of deletion or removal\cite{wiki_data_remanence} of data. 
The recovery of these bits may be due to the technique used to delete or the physical characteristics of the storage media. 
The problem of data remanence was first observed in magnetic media, when it was proven that data could be restored after several times of overwriting. 
\todo{Write something more.. general stuff about data remanence}

Earlier it has been shown that for some UV EPROM, EEPROM and Flash devices, information can still be recovered after 100 erase cycles\cite{data_remanence_flash}.


\subsection{Differential fault analysis}\label{sec:differential_fault_analysis}

Differential fault analysis is cryptanalysis based on interpreting fault outputs from a processor exposed to different kinds of environmental exposure like high voltage or current, strong elektromagnetic fields or as simple as high temprature. 
The processor might give fault outputs an adversary can use to find knowledge about the processor state or similar information.


Guo, et al (2012): Invariance-based concurrent error detection for advanced encryption standard\cite{dfa_guo}
\todo{Is this paper something to write about?}

Giraud (2003): DFA on AES\cite{dfa_aes}
\todo{Is this paper something to write about?}

\subsection{Electromagnetic attacks}\label{sec:electromagnetic_attacks}

An adversary can, if located within range, intercept electromagnetic signal emitted by all electronic devices. 
It is not possible to detect this attack, and therefore hard to prevent. 
One of the known attacks is extracting the signals from a VGA cable to reproduce the screen output.
PGP includes a secure viewer that prevents exactly this. 
However, it is said to be very difficult, thus expensive to extract the signals emitted by a VGA cable. 

NSA: TEMPEST (code name for project)
\todo{Write about TEMPEST}

\subsection{Power analysis}\label{sec:power_analysis}

Electronic devices leaks information about what they are processing through their power consumption.
This leakage can be used to extract secret information like keys used to encryption using differential power analysis (DPA). 

Simple power analysis (SPA) is simply interpreting the power consumption measurements during cryptographioc operations\cite{dpa_kocher}. 

DPA is a more complex way to interpret the power consumption. \todo{write something fancy about DPA}
Paul Kocher: Differential power analysis \cite{dpa_kocher}

\subsection{Timing attacks}\label{sec:timing_attacks}

Here goes related work to timing attacks

SSL Timing attack \cite{ssl_timing_attack}
Timing attack Diffie-Hellman \cite{timing_attack_kocher}

\section{Original research - Acoustic Cryptanalysis}\label{sec:original_research}

Original paper \cite{original_paper}

\subsection{Lab setup}\label{sec:lab_setup}
\subsection{Acustic leakage}\label{sec:acustic_leakage}
\subsection{RSA key extraction}\label{sec:rsa_key_extraction}
