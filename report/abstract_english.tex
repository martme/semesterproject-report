\pagestyle{empty}
\begin{abstract}
Computers emit low bandwidth acoustic signals, i.e. sounds audibly by the human ear and up to several hundred kHz. 
This phenomena exposes several computers for acoustic side-channel attacks by adversaries who wants to gain information about what operations the CPU is performing, and possibly extracting cryptographic keys.

In this paper we build a portable and a lab grade setup that we use to verify the existence of the acoustic side-channel found in numerous computers.

Our empirical observations clearly enhances the suspicion of the existence of interpretable acoustic emanations from the range 0-100kHz. 
We find it easy to distinguish between some CPU operations and generally the CPUs workload.
We also demonstrate that the acoustic fingerprints vary a lot depending on the processor, thus the signal can be indistinguishable on some newer CPUs. 


\section*{Acknowledgment}\label{acknowledgment}
Stig Frode Mjølsnes from the \gls{ITEM} at the \gls{ntnu} guided us through our work. 
\gls{ITEM} sponsored us with all of our equipment used in our research.
Tim Cato Netland from the \gls{IET} at the \gls{ntnu} recommended equipment for our portable setup and helped us assemble the lab grade setup.
We also thank numerous of people at the \gls{IET} that helped us to with our portable setup. 


\end{abstract}