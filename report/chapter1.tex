\chapter{Introduction}\label{chp:introduction} 

\section{Motivation}
In 2014, Genkin et al. presented their work on exploiting low-bandwidth acoustic emanations to extract a full 4096-bit RSA key~\cite{DBLP:conf/crypto/GenkinST14}.
This novel approach opens for a whole new attack vector that has unexplored potential, as the emanations are allegedly able to leak information about what is being executed on the CPU.
In this paper we will verify some of the claims presented by Genkin et al., by setting up and performing the experiments that are outlined, and possibly reproduce the results presented.


\section{Problem and Scope}
The main goal of this paper is to verify the existence of the acoustic side-channel, and reproduce some of the results of~\cite{DBLP:conf/crypto/GenkinST14}.
We build a similar setup to what was presented, and use it to record computers running in different states.
Then we analyze the recorded acoustic emanations, and attempt to relate the CPU operations to the acoustic emanations.

We diverge from the original project description in that we do not emphasize on applications with regard to cryptanalysis.
Rather we explore the extremes of CPU operations, to be able to verify the initial results of Genkin et al.
We rely strictly on empiric evaluation of our results during analysis, and use no statistical correlation models.
This is done despite the fact that it introduces some limitations to our exploration of the side channel, as the proposed attack on the vulnerable RSA implementation would require such a model.
However, the approach is deemed viable due to the results presented in~\cite{DBLP:conf/crypto/GenkinST14}, and it requires a simpler setup.
Thus RSA key extraction is beyond the scope of this paper.

Using this setup we will show that we are indeed successful in reproducing some of the results presented by Genkin et al., by distinguishing between different CPU activities based on the acoustic fingerprint and computer assisted empirical evaluation.

\section{Methodology}
We record the acoustic emanations of known execution patterns.
Then, we represent the acoustic signatures visually in frequency spectrograms, and empirically try to correlate the nature of the spectrograms with the execution patterns.

\section{Outline}
In this paper we will first present related work, and give a short presentation of the background of our work.
Then we will present our experimental setup, first in the form of the instrumentation; then in the form of the code that we execute on the computers during our experiments.
After this has been described in detail we will present our results, and discuss our findings.
