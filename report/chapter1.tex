\chapter{Introduction}\label{chp:introduction} 

\section{Motivation}
In 2014, Genkin et al. presented their work on exploiting low-bandwidth acoustic emanations to extract a full 4096 bit RSA key~\cite{DBLP:conf/crypto/GenkinST14}.
This novel approach opens for a whole new attack vector that has unexplored potential, as the emanations are allegedly able to leak information about what is being executed on the CPU.
In this paper we will verify some of the claims presented by Genkin et al., by setting up and performing the experiments that are outlined, and possibly reproduce the results presented.

\section{Problem and Scope}
The main goal of this paper is to verify the existence of the acoustic side channel, and reproduce some of the results of~\cite{DBLP:conf/crypto/GenkinST14}.
We will build a similar setup as to what is presented, and use it to record computer devices running in different states.
Then we will analyze the recorded acoustic emanations, and attempt to relate them to CPU operations and states of the computer to the acoustic emanations.

We will diverge from the original project description in which we will not emphasize on applications with regard to cryptanalysis.
Rather we will explore extremes of CPU operations, to be able to verify the initial results of Genkin et al.
We are relying strictly on empiric evaluation of our results during analysis, and are using no statistical correlation models.
This approach is deemed viable due to the results presented in~\cite{DBLP:conf/crypto/GenkinST14}, and leads to a more simple setup.
This is done despite the fact that it introduces some limitations in our exploration of the side channel, as the proposed attack on the vulnerable RSA implementation would require such a model.
Thus RSA key extraction is beyond the scope of this paper.


%Write about changes in scope. Redefine and explain why it has changed.
%\section{Limitations}

\section{Outline}
In this paper we will first present related work, and give a short presentation of the background of our work.
Then we will present our experimental setup, first in the form of the instrumentation; then in the form of the code that we execute on the computers during our experiments.
After this has been described in detail we will present our results, and discuss our findings.