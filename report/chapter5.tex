\chapter{Results}
\label{chp5:results} 
In this chapter we present our results we have gained from our empirical research. 
All executions that we present are described in~\autoref{chp4:predictable_execution} unless else specified. 
Generally we have three different executions; microinstructions described in~\autoref{chp4:sec:microinstructions}, CPU load described in~\autoref{chp4:sec:cpu_load} and decryption described in~\autoref{chp4:sec:decryption}. 
We are working with two different microphone configurations, the Knowles microphone and the Brüel\&Kjær microphone, which is explained more in detail respectively in~\autoref{chp3:sec:knowles_configuration} and~\autoref{chp3:sec:bruel_kjaer_configuration}.
First we present our result from the Knowles microphone.

After the Knowles results, we list the results that are produced with the Brüel\&Kjær 4939 configuration in accordance with the experimental plan in~\autoref{apx:experiment_plan}. 
These results are from the anechoic chamber and the associated results from an office environment.  
We will first present the results gained from the microinstructions, followed by the results from CPU load and decryption. 

%===================================================================================================================
%                                               Knowles microphone
%===================================================================================================================
\section{Results from Knowles}\label{chp5:sec:knowles_results}
The results presented under this section are all gained from the Knowles microphone and different types of setups.
The recordings are done in a try-and-fail approach rather than according to an experimental plan. 

\subsection{Results from Lenovo T60p - decryption}\label{chp5:subsec:t60p_knowles_results_decryption}
In~\autoref{fig:T60p-knowles-decrypt-ips} we ran the decryption. 
In~\autoref{fig:T60p-knowles-decrypt-ips-2} we ran a different decryption setup, described in~\autoref{lst:decryption_loop}.

\begin{lstlisting}[
language=BASH, 
caption={Decryption of an image file.}, 
label={lst:decryption_loop}]
sleep 1             
gpg --output tmp/image_de.png --decrypt img.gpg
sleep 1
gpg --output tmp/image_de2.png --decrypt img.gpg
sleep 1
gpg --output tmp/image_de3.png --decrypt img.gpg
gpg --output tmp/image_de4.png --decrypt img.gpg    
gpg --output tmp/image_de5.png --decrypt img.gpg
gpg --output tmp/image_de6.png --decrypt img.gpg
\end{lstlisting}

%================
% T60p deryption
%================
\begin{figure}[ht]
    \begin{subfigure}{0.5\textwidth}
        \centering
        \includegraphics[width=1\linewidth]{T60p-knowles-decrypt-ips_description.png}
        \caption{}
        \label{fig:T60p-knowles-decrypt-ips}
    \end{subfigure}
    \begin{subfigure}{0.5\textwidth}
        \centering
        \includegraphics[width=1\linewidth]{T60p-knowles-decrypt-ips-2_description.png}
        \caption{}
        \label{fig:T60p-knowles-decrypt-ips-2}
    \end{subfigure}
    \caption{Acoustic recording (Vertical axis: 7 sec. Horizontal axis: 0-100kHz) of the Lenovo T60p when running decryption.
    Both recordings was made in an office environment using the Knowles configuration and running on battery power. }
    \label{fig:T60p-knowles-decrypt-ips}
\end{figure}

%================
% T60p micro
%================
\subsection{Results from Lenovo T60p - microinstructions}\label{chp5:subsec:t60p_knowles_results_micro}
The results presented in~\autoref{fig:T60p-knowles-micro-ips-0} has been produced when running microinstructions.
The plotting of this result is achieved by using a different approach with regard to the signal processing. 
Since we are not required to transform the sound file back to the time domain, we are able to sample overlapping windows.
In the case of this plot, we are taking the fourier transform of windows of size 2048 samples, but only moving the sliding window \({W/2}\) samples.
Additionally, the plot shows the 10th logarithm of the frequency responses, and only values between the median value (min) and 0 (max) are indicated with the colors (in a linear scale, as in the other plots). 
\begin{figure}[ht]
    \centering
    \includegraphics[width=0.6\linewidth]{T60p-knowles-micro-ips-0_description.png}
    \caption{Acoustic recording (Vertical axis: 5 sec. Horizontal axis: 0-100kHz) of the Lenovo T60p when running microinstructions. The recording was made in an office environment using the Knowles configuration and running on battery power. }
    \label{fig:T60p-knowles-micro-ips-0}
\end{figure}
%================
% T60p CPU load
%================
\subsection{Results from Lenovo T60p - CPU load}\label{chp5:subsec:t60p_knowles_results_cpuload}
The result in~\autoref{fig:T60p-knowles-cpuload-ips-0} has been produced running full CPU load. The plotting of this result is achieved with the same approach as described in~\autoref{chp5:subsec:t60p_knowles_results_micro}. 
In~\autoref{fig:T60p-knowles-cpuload-ips-0} is the result gained from running a full CPU load in the Lenovo T60p computer. 
\begin{figure}[ht]
    \centering
    \includegraphics[width=0.6\linewidth]{T60p-knowles-cpuload-ips-0_description.png}
    \caption{Acoustic recording (Vertical axis: 5 sec. Horizontal axis: 0-100kHz) of the Lenovo T60p when running a full CPU load. The recording was made in an office environment using the Knowles configuration and running on battery power. }
    \label{fig:T60p-knowles-cpuload-ips-0}
\end{figure}

%===================================================================================================================
%                                               Brüel&Kjær microphone
%===================================================================================================================
%=============
% T60p MICRO
%=============
\section{Results from Brüel\&Kjær}\label{chp5:sec:bk_results}
In this section we present the results gained from the Brüel\&Kjær configuration. 
In contrast to the results presented in~\autoref{chp5:sec:knowles_results}, these experiments are carefully done according to the experimental plan described in in~\autoref{apx:experiment_plan}. 

\subsection{Results from Lenovo T60p - microinstructions}\label{chp5:subsec:t60p_bk_results_micro}
The~\autoref{fig:T60p-ekkofritt-bk-micro-eps-1} and the~\autoref{fig:T60p-ekkofritt-bk-micro-ips-3} is the result from doing the microinstructions in the anechoic chamber on the Lenovo T60p computer. The~\autoref{fig:T60p-bk-micro-ips-0} is the same recordings as~\autoref{fig:T60p-ekkofritt-bk-micro-eps-1} and~\autoref{fig:T60p-ekkofritt-bk-micro-ips-3}, but in an office environment.
%===================================================
% T60p Anechoic chamber & normal room MICRO
%===================================================

\begin{figure}[ht]
    \centering
    \includegraphics[width=1\linewidth]{T60p-ekkofritt-bk-micro-eps-1_description.png}
    \caption{Acoustic recordings (Vertical axis: 6 sec. Horizontal axis: 0-100kHz) of the Lenovo T60p when running microinstructions using the Brüel\&Kjær 4939 configuration. The recording was made in an anechoic chamber.}
    \label{fig:T60p-ekkofritt-bk-micro-eps-1}
\end{figure}

% \begin{figure}[ht]
% 	\begin{subfigure}{0.32\textwidth}
% 	    \centering
% 	    \includegraphics[width=1\linewidth]{T60p-ekkofritt-bk-micro-eps-1_description.png}
% 	    \caption{With power adopter}
% 	    \label{fig:T60p-ekkofritt-bk-micro-eps-1}
%     \end{subfigure}
%     \begin{subfigure}{0.32\textwidth}
% 	    \centering
%     	\includegraphics[width=1\linewidth]{T60p-ekkofritt-bk-micro-ips-3.png}
%     	\caption{Running on battery power}
%     	\label{fig:T60p-ekkofritt-bk-micro-ips-3}
%     \end{subfigure}
%     \begin{subfigure}[ht]{0.32\textwidth}
%         \centering
%         \includegraphics[width=1\linewidth]{T60p-bk-micro-ips-0.png}
%         \caption{}
%         \label{fig:T60p-bk-micro-ips-0}
%     \end{subfigure}
%     \caption{Acoustic recordings (Vertical axis: 6 sec. Horizontal axis: 0-100kHz) of the Lenovo T60p when running microinstructions. All three using the Brüel\&Kjær 4939 configuration. ~\autoref{fig:T60p-ekkofritt-bk-micro-ips-3} and~\autoref{fig:T60p-ekkofritt-bk-micro-eps-1} recordings was made in an anechoic chamber. The~\autoref{fig:T60p-bk-micro-ips-0} is done in a normal office environment. }
% 	\label{fig:T60p-ekkofritt-bk-micro}
% \end{figure}

%==============
% T60p CPULOAD
%==============
\subsection{Results from Lenovo T60p - CPU load}\label{chp5:subsec:t60p_bk_results_cpuload}
The~\autoref{fig:T60p-ekkofritt-bk-cpuload-eps-4} and the~\autoref{fig:T60p-ekkofritt-bk-cpuload-ips-0} is the results from the running CPU load on the Lenovo T60p in the anechoic chamber. 
%===============================
% T60p Anechoic chamber CPULOAD
%===============================

\begin{figure}[ht]
	\begin{subfigure}{0.5\textwidth}
	    \centering
	    \includegraphics[width=1\linewidth]{T60p-ekkofritt-bk-cpuload-eps-4_description.png}
	    \caption{With power adopter}
	    \label{fig:T60p-ekkofritt-bk-cpuload-eps-4}
    \end{subfigure}
    \begin{subfigure}{0.5\textwidth}
	    \centering
	    \includegraphics[width=1\linewidth]{T60p-ekkofritt-bk-cpuload-ips-0_description.png}
	    \caption{Running on battery power}
	    \label{fig:T60p-ekkofritt-bk-cpuload-ips-0}
    \end{subfigure}
    \caption{Acoustic recording (Vertical axis: 10 sec. Horizontal axis: 0-100kHz) of the Lenovo T60p when running a full CPU load. The recording was made in an anechoic chamber using the Brüel\&Kjær 4939 configuration.}
	\label{fig:T60p-ekkofritt-bk-cpuload}
\end{figure}
%==========================
% T60p Normal room CPULOAD
%==========================
The~\autoref{fig:T60p-bk-cpuload-eps-1-1a} and the~\autoref{fig:T60p-bk-cpuload-ips-0-1b} is the results from the running CPU load on the Lenovo T60p in an office environment. 
\begin{figure}[ht]
	\begin{subfigure}{0.5\textwidth}
	    \centering
	    \includegraphics[width=1\linewidth]{T60p-bk-cpuload-eps-1_description.png}
	    \caption{With power adopter}
	    \label{fig:T60p-bk-cpuload-eps-1-1a}
    \end{subfigure}
    \begin{subfigure}{0.5\textwidth}
	    \centering
	    \includegraphics[width=1\linewidth]{T60p-bk-cpuload-ips-0_description.png}
	    \caption{Running on battery power}
	    \label{fig:T60p-bk-cpuload-ips-0-1b}
    \end{subfigure}
    \caption{Acoustic recording (Vertical axis: 10 sec. Horizontal axis: 0-100kHz) of the Lenovo T60p when running a full CPU load described in~\autoref{chp4:sec:cpu_load}. The recordings was made using the Brüel\&Kjær 4939 configuration. }
	\label{fig:T60p-bk-cpuload}
\end{figure}


%==============
% D430 CPULOAD
%==============
\subsection{Result from Dell D430 - CPU load}\label{chp5:subsec:d430_bk_results_cpuload}

%================================
% D430 Anechoic chamber  CPULOAD
%================================
The~\autoref{fig:D430-ekkofritt-bk-cpuload-eps-1} is the result from the running CPU load on the Dell 430 in the anechoic chamber. 
\begin{figure}[ht]
    \centering
    \includegraphics[width=0.7\linewidth]{D430-ekkofritt-bk-cpuload-eps-1_description.png}
    \caption{Acoustic recording (Vertical axis: 10 sec. Horizontal axis: 0-100kHz) of the Dell D430 when running a full CPU load. The recording was made in an anechoic chamber using the Brüel\&Kjær 4939 configuration. }
    \label{fig:D430-ekkofritt-bk-cpuload-eps-1}
\end{figure}


%=================
% T60p Decryption
%=================
\subsection{Result from Lenovo T60p - Decryption}\label{chp5:subsec:t60p_bk_results_decryption}

%===============================
% T60p Anechoic chamber DECRYPT
%===============================
The~\autoref{fig:T60p-ekkofritt-bk-decrypt-ips-3} is the result from the running decryption on the Lenovo T60p in the anechoic chamber. 
\begin{figure}[ht]
    \centering
    \includegraphics[width=0.7\linewidth]{T60p-ekkofritt-bk-decrypt-ips-3_description.png}
    \caption{Acoustic recording (Vertical axis: 6 sec. Horizontal axis: 0-100kHz) of the Lenovo T60p when running a decryption. The recording was made in an anechoic chamber using the Brüel\&Kjær 4939 configuration. }
    \label{fig:T60p-ekkofritt-bk-decrypt-ips-3}
\end{figure}

%=================
% D430 Decryption
%=================
\subsection{Result from Dell D430 - Decryption}\label{chp5:subsec:d430_bk_results_cpuload}

%================================
% D430 Anechoic chamber  DECRYPT
%================================
The~\autoref{fig:D430-ekkofritt-bk-decrypt-eps-0} is the result from the running decryption on the Dell 430 in the anechoic chamber. 
\begin{figure}[ht]
    \centering
    \includegraphics[width=0.7\linewidth]{D430-ekkofritt-bk-decrypt-eps-0_description.png}
    \caption{Acoustic recording (Vertical axis: 6 sec. Horizontal axis: 0-100kHz) of the Dell D430 when running a decryption. The recording was made in an anechoic chamber using the Brüel\&Kjær 4939 configuration.}
    \label{fig:D430-ekkofritt-bk-decrypt-eps-0}
\end{figure}


%=============================
% D430 Anechoic chamber  IDLE
%=============================
%The~\autoref{fig:D430-ekkofritt-bk-idle-eps-1} is the result from the running idle on the Dell 430 in the anechoic chamber. 
%\begin{figure}[ht]
%    \centering
%    \includegraphics[width=0.7\linewidth]{D430-ekkofritt-bk-idle-eps-1.png}
%    \caption{Acoustic recording (Vertical axis: 5 sec. Horizontal axis: 0-100kHz) of the Dell D430 when idle. The recording was made in an anechoic chamber using the Brüel\&Kjær 4939 configuration specified in~\autoref{chp3:sec:bruel_kjaer_configuration}. }
%    \label{fig:D430-ekkofritt-bk-idle-eps-1}
%\end{figure}

%=====================
% Raspberry PI
%=====================
\subsection{Results from Raspberry PI}\label{chp5:subsec:rb_bk_results}
All results from the Raspberry PI are inconclusive. 
I.e. we are not able to interpret any of the recordings and relate them to the operations we performed.
