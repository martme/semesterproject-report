\chapter{Code Used in Analysis}\label{apx:code}

\section{Processing Sound Files}
The following program written in C was used to process the sound files recorded in our experiments. 
The reason for writing the program this way is because we were more comfortable with writing in C than using i.e. Matlab.
Additionally, we think this code is very adoptable, and can easily be adjusted to process real time recordings.

\section{Usage}
The output from \autoref{lst:main.c} is piped to \autoref{lst:plot.py}, to produce a plot representing the frequency response for the duration of the recording. 


\section{Signal Processing in C}
The signal processing is done with \autoref{lst:main.c}, a routine written in C.
The frequency responses resulting from each Fourier transformed window is passed to the standard output, as well as some meta data.

\begin{lstinputlisting}[language=C, caption={main.c - Compute Power Spectrum of a Sound File}, label={lst:main.c}]{code/main.c}
\end{lstinputlisting}


\section{Plotting in Python}
Plotting is done using the Python script given in \autoref{lst:plot.py}.
The data is read from standard input, i.e. the output from the signal processing routine.
Simple operations, such as applying a logarithmic scale to the spectra, and setting thresholds for the plot is done by tuning parameters in this script.

\begin{lstinputlisting}[language=Python, caption={plot.py - Plot frequency spectra on time axis}, label={lst:plot.py}]{code/plot.py}
\end{lstinputlisting}
