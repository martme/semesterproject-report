\chapter{Conclusion and Future Work}\label{chp7:conclusion}

\section{Conclusion}
In this paper we have conducted independent experiments with an aim on verifying the claims of Genkin et al. 
The experiments have touched a part of the results presented in~\cite{DBLP:conf/crypto/GenkinST14}, as they have aimed to verify the bare existence of the acoustic side channel as a viable medium to extract information leakage.

Our research show that we are able to support the claims that the acoustic side channel conveys enough information to distinguish between different modes of low-level CPU operations.
We display that the \gls{SNR} in some cases is strong enough to easily spot such differences using simple plots of the frequency spectra.

The most significant result obtained is the clear display of difference in the acoustic fingerprint of the MEM operation and looping other microinstructions.
Differences can even be observed at frequencies in the audible range, which supports the claim made by Genkin et al., that a mobile phone's microphone can be used to extract viable information from the side channel.
Additionally, this suggests that low-cost equipment can be sufficient for information extraction over this side-channel, something which is supported by our results using our experimental recording setup as described in~\autoref{chp3:sec:knowles_configuration}.

Results presented in this paper support the claims by Genkin et al.; low-bandwidth acoustic emanations from computers carry information about the internal state of the \gls{CPU}.
Moreover, extracting this information can be done with a relatively cheap setup, under sub-optimal conditions.
Our experiments show that recording in an anechoic chamber is in no way a necessity, and that the \gls{SNR} while recording in a room full of other people and different computer equipment is sufficient to clearly see differences in the acoustic fingerprints of distinct \gls{CPU} operations.



\section{Future Work}\label{chp7:sec:future_work}
An assumption taken in our work is that the emanations observed during the experiments are in fact acoustic.
This question is discussed by Genkin et al., and they have conducted experiments to support their claim.
We have not reproduced any of these experiments, as we have merely looked at information leakage from CPUs operating under very specific conditions.
Future work will aim to support the theory that the emanations are acoustic.
Additionally, understanding the phenomenons causing the emanations, and identifying the source is a subject to research in the future.




\begin{enumerate}
	\item ADD loops of different lengths
	\item Statistical models for correlation
	\item Real time analysis
	\item Experiments on the proposed attack against RSA
	\item Listening to GPU for reproducing display
\end{enumerate}