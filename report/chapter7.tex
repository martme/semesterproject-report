\chapter{Conclusion and Future Work}\label{chp7:conclusion}

\section{Conclusion}

An assumption taken in our work is that the emanations observed during the experiments are in fact acoustic.
This question is discussed by Genkin et al., and they have conducted experiments to support their claim stating that this is true.
All our research rely on the validity of this claim.

In this paper we have conducted independent experiments with an aim on verifying the claims of Genkin et al. 
The experiments have touched a part of the results presented in~\cite{DBLP:conf/crypto/GenkinST14}, as they have aimed to verify the bare existence of the acoustic side channel as a viable medium to extract information leakage.

Our research show that we are able to support the claims that the acoustic side channel conveys enough information to distinguish between different low-level \gls{CPU} operations.
We display that the \gls{SNR} in some cases is strong enough to easily spot such differences using simple plots of the frequency spectra.

The most significant result obtained is the clear display of difference in the acoustic fingerprint of the \texttt{MEM} operation and looping other microinstructions.
Differences can even be observed at frequencies in the audible range, which supports the claim made by Genkin et al., that a mobile phone's microphone can be used to extract viable information from the side channel.
Additionally, this suggests that low-cost equipment can be sufficient for information extraction over this side-channel, something which is supported by our results using our portable recording setup as described in~\autoref{chp3:sec:knowles_configuration}.

Results presented in this paper support the claims by Genkin et al.; low-bandwidth acoustic emanations from computers carry information about the internal state of the \gls{CPU}.
Moreover, extracting this information can be done with a relatively cheap setup, under sub-optimal conditions.
Our experiments show that recording in an anechoic chamber is in no way a necessity, and that the \gls{SNR} while recording in a room full of other people and different computer equipment is sufficient to clearly see differences in the acoustic fingerprint of distinct \gls{CPU} operations.

\section{Future Work}\label{chp7:sec:future_work}
We have not reproduced any of the experiments to verify the claim that the emanations observed are in fact acoustic.
Future work should aim to support this theory.
Understanding the phenomenons causing the emanations, and identifying the source should also be subject to future research.

In this paper we have have merely looked at information leakage from CPUs operating under very specific extreme conditions.
Building statistical correlation models to look at more subtle changes of acoustic signatures should also be a goal for future work, as the work presented in this paper is limited by the methodology chosen.
This approach, combined with real-time analysis, would allow for experiments with suggested applications in cryptanalysis, such as the proposed side-channel attack on the vulnerable RSA implementation.
