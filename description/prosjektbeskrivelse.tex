\documentclass[a4paper,11pt]{article}
\usepackage{graphicx}
\usepackage{times}
\usepackage[utf8]{inputenc} 
\usepackage[norsk]{babel} 
\topmargin=5.mm
\oddsidemargin=0.mm
\evensidemargin=0.mm
\textheight=220.mm 
\textwidth=170.mm
\parindent 0pt
\parskip .5em
\renewcommand{\arraystretch}{1.5}

\begin{document}
\sffamily
\begin{titlepage}
\begin{center}
\textsc{NORGES TEKNISK-NATURVITENSKAPELIGE UNIVERSITET\\
FAKULTET FOR INFORMASJONSTEKNOLOGI, MATEMATIKK OG ELEKTROTEKNIKK} \\
\vspace{0.5cm} 
\includegraphics[scale=0.5]{NTNU-logo} \\
\vspace{1.0cm}
{\Huge{PROSJEKTOPPGAVE}}
\vspace{1.0cm}
\end{center}

\begin{tabular}{@{}p{5cm}l}
Kandidatens navn:	& Haakon Garseg Mørk og Martin Kirkholt Melhus\\
Emne:			& TTM4531, fordypningsprosjekt\\
Oppgavens tittel: 	& Side-channel attacks on cryptographic implementations \\
Oppgavens beskrivelse: 	& \\
\end{tabular}

In late 2013, a team of researchers managed to extract a full 4096 bit RSA key using low-bandwidth
acoustic noise as a side channel. In 2014, Daniel Genkin, Adi Shamir and Eran Tromer presented
their work in the paper titled RSA Key Extraction via Low-Bandwidth Acoustic Cryptanalysis.

In this project, we aim to verify the results from the original research. We will verify the existence of
the acoustic side channel, and analyse the acoustic fingerprint resulting from a computer executing
microinstructions. We will do this by building our own experimental setup, and verify its capability of
identifying these fingerprints.

After verifying the existence of the side channel we will try to apply some of the techniques described in the original research and exploit the information leakage to obtain information about ongoing processes in various target computers.

\begin{tabular}{@{}p{5cm}l}
Utf\o{}rt ved:	& Institutt for telematikk \\
Ideinnehaver:	& Sondre Rønjom \\
Veileder:	& Markku-Juhani O. Saarinen \\
Fagl\ae{}rer: 	& Stig F. Mjølsnes \\
\end{tabular}

\end{titlepage}
\end{document}
